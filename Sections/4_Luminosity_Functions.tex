\section{Luminosity Functions} \label{Sec: Luminosity Functions}
\subsection{Vmax} \label{Sec: Vmax}

\begin{itemize}
    \item \textcolor{red}{However, my major concern is about the result of the LFs shown in Figure 3: how can one of the components be much higher in normalisation than the total LF? There are several incongruences in the LFs, also at the bright-end, and I would like the authors to critically check their analysis and results to find the reason of such inconsistencies before resubmitting the paper.}
    \item \textcolor{Green}{Clarify that ZFOURGE total is not CIGALE total}
    \vspace{0.5cm}

    
    \item \textcolor{red}{The concern regarding the functional forms is due to the fact that is difficult to compare the main parameters obtained with different forms: even if they are called in the same way, they might not be directly comparable. I therefore would like the authors to use the same form for all the components and to compare the resulting parameters only to works in the literature using the same form.}
    \item \textcolor{green}{We've explicittly done schecther for SF and saunders for AGN as is often done in the literature. x, y, z, use schechter for sf, a, b, c, use saunders for AGN. This was done to better compare to the literature.}
    
    \item \textcolor{red}{Please, say explicitly which values of the cosmological parameters are considered. Not all readers necessarily use python and are due to know what is contained in the FlatLambdaCDM package).}
    \item \textcolor{green}{Cosmology is explictiy state in the introduction.}
    
    \item \textcolor{red}{"Sources with Dmax values that do not extend to the end of the redshift bin are removed to avoid bias from sources incompletely sampled within the bin." -> As mentioned before, this avoids a small incompleteness (or rather inhomogeneity in the redshift bin), but introduces a larges one, because excluding sources within the bin lower the real (or best possible) volume density of sources. I would therefore suggest the author to keep these sources and recompute the LFs.}
    \item \textcolor{green}{I have changed this, update later}
    
    \item \textcolor{red}{The considered error in the LF is simply the Poissonian value. Usually simulations considering also the possible sources of uncertainties in zphot, and Lir should be run, providing more realistic (and larger) error-bars. If there are reasons to believe that these uncertainties are much smaller than the poissonian ones (and can be neglected), the authors should state explicitly in the text and comment on that.}
    \item \textcolor{green}{I have done this now}
    
    \item \textcolor{red}{I urge the authors to discuss about the sample incompleteness and how they deal with it, as the exclusion of different classes of sources have certainly introduced different levels of incompleteness, which should be accounted for in the LF calculation.}
    \item \textcolor{green}{Potentially bias associated with my conservate approach. How it affects normalisation/gright/dim ends}
\end{itemize}

To estimate the LF from our data, we utilise the $1/V_{max}$ method \citep{schmidt_space_1968}. The $1/V_{max}$ method is well-suited for surveys like ZFOURGE, as it does not assume any specific shape for the LF and can easily accommodate galaxies observed across varying depths. It accounts for the maximum observable volume of each galaxy and is given by equation \ref{EQ: 1/Vmax}:

\begin{equation} \label{EQ: 1/Vmax}
    \phi(L,z) = \frac{1}{\Delta \log L}\sum_{i=1}^N \frac{1}{V_{max,i}}
\end{equation}

Where $V_{max}$ represents the maximum co-moving volume of the $i$-th source and $\Delta$ log(L) is the width of the luminosity bin. In practice, to observe the evolution of the LF through cosmic time, the maximum observable volume ($V_{max}$) is calculated for each redshift bin where the upper and lower bounds of the redshift bin limit the volume. Additionally, redshift bins are split into luminosity bins to observe the number density evolution across the different classes of luminosity such as LIRGs (10$^{11} < L_{IR} < 10^{12}\ L_{\odot}$) and ULIRGs ($L_{IR} > 10^{12}\ L_{\odot}$). $V_{max}$ of each galaxy is calculated by taking the maximum comoving volume of the redshift bin the galaxy resides in and subtracting the comoving volume at the beginning of the redshift bin (equation \ref{EQ: Vmax}). We account for the survey area of ZFOURGE (0.1111 degrees$^2$), which normalises the volume probed across the sky (41,253 degrees$^2$). 

\begin{figure*}
    \centering
    \includegraphics[width=\textwidth]{Figures/LF.png}
    \caption{The luminosity functions of major galaxy populations in ZFOURGE and CIGALE calculated using the Vmax method. The dark blue triangles present the ZFOURGE bolometric IR (8-1000$\mu$m) LF. The CIGALE SF and AGN LFs are the green stars and red squares, respectively. The blue and red dashed lines show the best fit Saunders function \citep{saunders_60-mum_1990} to the ZFOURGE total and CIGALE AGN LF, respectively. The green dashed line shows the best-fit Schechter function \citep{schechter_analytic_1976} to the CIGALE SF LF. The shaded regions represent the $1\sigma$ functional fit errors. The luminosity completeness limit of each redshift bin is where we stop displaying fainter $\phi$ values. Where possible, comparable literature results are also shown. The local \cite{sanders_iras_2003} luminosity function is shown across all redshift bins as the grey dashed line. \cite{rodighiero_mid-_2010} is shown as gold filled stars from $0 < z < 2.5$, \cite{gruppioni_herschel_2013} as purple crosses from $0 < z < 4.2$, \cite{symeonidis_agn_2021} AGN as pink hexagons, \cite{thorne_deep_2022} AGN as maroon upside-down triangles, \cite{delvecchio_tracing_2014} AGN as indigo diamonds. Differing redshift bins are colour-labelled accordingly. Our ZFOURGE total results are consistent with various sources across redshift bins in the literature \citep{caputi_infrared_2007, huang_local_2007, fu_decomposing_2010}}
    \label{Fig: Bolometric IR LF}
\end{figure*}

\begin{equation}
    \label{EQ: Vmax}
    V_{max,i} = \frac{4}{3} \pi \left(D_{max}^3 - D_{min}^3\right) \times \frac{A}{41,253}
\end{equation}

We calculate the maximum ($D_{max}$) and minimum ($D_{min}$) comoving distances for all sources within each redshift bin using the \texttt{FlatLambdaCDM} model from the \texttt{Astropy} Python package \citep{astropy_collaboration_astropy_2022}. These calculations are performed for sources above the luminosity-completeness limits (coloured sources of figure \ref{Fig: ZF Lum vs z}). We limit each luminosity bin to a minimum of five sources, or else the luminosity bin is discarded. 

Sources with $D_{max}$ values that do not extend to the end of the redshift bin are removed to avoid bias from sources incompletely sampled within the bin. Different redshift bins have different volumes, and each luminosity bin has a different number density $\phi$. $D_{min}$ and $D_{max}$ are the comoving distances at the beginning and end of the redshift bin, respectively, for all galaxies. The relative LF number density $1\sigma$ error values are calculated with:

\begin{equation} \label{EQ: Vmax Error}
    \phi(L,z) = \frac{1}{\Delta \log L}\sqrt{\sum_i \frac{1}{V_{max}^2}}
\end{equation}

\begin{table*}
    \begin{center}
    \caption{ZFOURGE bolometric IR (8-1000$\mu$m) LF $\phi$ values.}
    \label{Tab: ZF LF}
    \begin{tabular}{@{}ccccccc@{}}
        \toprule
        $\log_{10}(L_{IR}/L_{\odot})$ & 0.00 $\leq z <$ 0.30 & 0.30 $\leq z <$ 0.45 & 0.45 $\leq z <$ 0.60 & 0.60 $\leq z <$ 0.80 & 0.80 $\leq z <$ 1.00 & 1.00 $\leq z <$ 1.20 \\
        \hline
         8.50 --  8.75 & -1.58 $\pm$ 0.04 & - & - & - & - & - \\
         8.75 --  9.00 & -1.70 $\pm$ 0.04 & - & - & - & - & - \\
         9.00 --  9.25 & -1.83 $\pm$ 0.05 & -1.64 $\pm$ 0.03 & - & - & - & - \\
         9.25 --  9.50 & -2.15 $\pm$ 0.07 & -1.79 $\pm$ 0.03 & -1.83 $\pm$ 0.03 & - & - & - \\
         9.50 --  9.75 & -2.19 $\pm$ 0.08 & -2.05 $\pm$ 0.05 & -1.89 $\pm$ 0.03 & -1.83 $\pm$ 0.02 & - & - \\
         9.75 -- 10.00 & -2.54 $\pm$ 0.12 & -2.40 $\pm$ 0.07 & -2.01 $\pm$ 0.03 & -1.83 $\pm$ 0.02 & -2.00 $\pm$ 0.02 & - \\
        10.00 -- 10.25 & -2.45 $\pm$ 0.11 & -2.38 $\pm$ 0.07 & -2.30 $\pm$ 0.05 & -1.98 $\pm$ 0.02 & -2.00 $\pm$ 0.02 & -2.21 $\pm$ 0.03 \\
        10.25 -- 10.50 & -2.64 $\pm$ 0.13 & -2.44 $\pm$ 0.07 & -2.37 $\pm$ 0.05 & -2.17 $\pm$ 0.03 & -2.09 $\pm$ 0.02 & -2.17 $\pm$ 0.02 \\
        10.50 -- 10.75 & -2.90 $\pm$ 0.18 & -2.60 $\pm$ 0.09 & -2.62 $\pm$ 0.07 & -2.34 $\pm$ 0.04 & -2.31 $\pm$ 0.03 & -2.41 $\pm$ 0.03 \\
        10.75 -- 11.00 & -                & -2.98 $\pm$ 0.14 & -2.74 $\pm$ 0.08 & -2.43 $\pm$ 0.04 & -2.52 $\pm$ 0.04 & -2.56 $\pm$ 0.04 \\
        11.00 -- 11.25 & -                & -                & -3.09 $\pm$ 0.12 & -2.71 $\pm$ 0.06 & -2.58 $\pm$ 0.04 & -2.75 $\pm$ 0.05 \\
        11.25 -- 11.50 & -                & -                & -3.50 $\pm$ 0.19 & -2.89 $\pm$ 0.07 & -2.85 $\pm$ 0.06 & -2.82 $\pm$ 0.05 \\
        11.50 -- 11.75 & -                & -                & -                & -3.34 $\pm$ 0.12 & -2.98 $\pm$ 0.07 & -3.20 $\pm$ 0.08 \\
        11.75 -- 12.00 & -                & -                & -                & -                & -                & -3.61 $\pm$ 0.13 \\
        \hline
        $\log_{10}(L_{IR}/L_{\odot})$ & 1.20 $\leq z <$ 1.70 & 1.70 $\leq z <$ 2.00 & 2.00 $\leq z <$ 2.50 & 2.50 $\leq z <$ 3.00 & 3.00 $\leq z <$ 4.20 & 4.20 $\leq z <$ 6.00  \\
        \hline
        10.50 -- 10.75 & -2.33 $\pm$ 0.02 & -2.47 $\pm$ 0.02 & - & - & - & - \\
        10.75 -- 11.00 & -2.39 $\pm$ 0.02 & -2.43 $\pm$ 0.02 & -2.54 $\pm$ 0.02 & - & - & - \\
        11.00 -- 11.25 & -2.52 $\pm$ 0.02 & -2.59 $\pm$ 0.03 & -2.58 $\pm$ 0.02 & -2.77 $\pm$ 0.03 & - & - \\
        11.25 -- 11.50 & -2.75 $\pm$ 0.03 & -2.71 $\pm$ 0.03 & -2.73 $\pm$ 0.02 & -2.70 $\pm$ 0.02 & - & - \\
        11.50 -- 11.75 & -2.91 $\pm$ 0.03 & -2.84 $\pm$ 0.04 & -2.97 $\pm$ 0.03 & -2.86 $\pm$ 0.03 & -3.02 $\pm$ 0.02 & - \\
        11.75 -- 12.00 & -3.21 $\pm$ 0.05 & -2.96 $\pm$ 0.04 & -3.03 $\pm$ 0.03 & -3.07 $\pm$ 0.04 & -2.96 $\pm$ 0.02 & -3.97 $\pm$ 0.06 \\
        12.00 -- 12.25 & -3.59 $\pm$ 0.07 & -3.03 $\pm$ 0.05 & -3.33 $\pm$ 0.05 & -3.35 $\pm$ 0.05 & -3.06 $\pm$ 0.02 & -3.71 $\pm$ 0.05 \\
        12.25 -- 12.50 & -4.02 $\pm$ 0.12 & -3.43 $\pm$ 0.07 & -3.62 $\pm$ 0.07 & -3.63 $\pm$ 0.07 & -3.33 $\pm$ 0.03 & -3.71 $\pm$ 0.04 \\
        12.50 -- 12.75 & -                & -3.87 $\pm$ 0.12 & -4.18 $\pm$ 0.13 & -4.11 $\pm$ 0.12 & -3.74 $\pm$ 0.05 & -3.97 $\pm$ 0.06 \\
        12.75 -- 13.00 & -                & -                & -                & -                & -4.06 $\pm$ 0.08 & -4.18 $\pm$ 0.08 \\
        13.00 -- 13.25 & -                & -                & -                & -                & -4.53 $\pm$ 0.13 & -4.84 $\pm$ 0.16
        \botrule
    \end{tabular}
    \end{center}
    \begin{tabnote}
        {\textbf{Note}: Luminosity bin $\phi$ values are centred.}\tnp
    \end{tabnote}
\end{table*}

\begin{table*}
    \begin{center}
    \caption{CIGALE AGN LF $\phi$ values.}
    \label{Tab: CG AGN LF}
    \begin{tabular}{@{}ccccccc@{}}
        \toprule
        $\log_{10}(L_{IR}/L_{\odot})$ & 0.00 $\leq z <$ 0.30 & 0.30 $\leq z <$ 0.45 & 0.45 $\leq z <$ 0.60 & 0.60 $\leq z <$ 0.80 & 0.80 $\leq z <$ 1.00 & 1.00 $\leq z <$ 1.20 \\
        \hline
         8.00 --  8.25 & -2.57 $\pm$ 0.12 & -                & -                & -                & -                & - \\
         8.25 --  8.50 & -2.64 $\pm$ 0.13 & -                & -                & -                & -                & - \\
         8.50 --  8.75 & -2.84 $\pm$ 0.16 & -2.64 $\pm$ 0.09 & -                & -                & -                & - \\
         8.75 --  9.00 & -2.74 $\pm$ 0.14 & -2.78 $\pm$ 0.11 & -2.58 $\pm$ 0.07 & -                & -                & - \\
         9.00 --  9.25 & -2.60 $\pm$ 0.13 & -2.90 $\pm$ 0.13 & -2.58 $\pm$ 0.07 & -2.48 $\pm$ 0.04 & -                & - \\
         9.25 --  9.50 & -2.84 $\pm$ 0.16 & -2.98 $\pm$ 0.14 & -2.74 $\pm$ 0.08 & -2.69 $\pm$ 0.05 & -2.46 $\pm$ 0.04 & - \\
         9.50 --  9.75 & -                & -2.90 $\pm$ 0.13 & -2.57 $\pm$ 0.07 & -2.76 $\pm$ 0.06 & -2.58 $\pm$ 0.04 & -2.67 $\pm$ 0.04 \\
         9.75 -- 10.00 & -                & -2.81 $\pm$ 0.11 & -2.79 $\pm$ 0.09 & -2.75 $\pm$ 0.06 & -2.91 $\pm$ 0.06 & -2.79 $\pm$ 0.05 \\
        10.00 -- 10.25 & -                & -2.78 $\pm$ 0.11 & -2.92 $\pm$ 0.10 & -2.80 $\pm$ 0.06 & -2.94 $\pm$ 0.06 & -2.98 $\pm$ 0.06 \\
        10.25 -- 10.50 & -                & -                & -3.12 $\pm$ 0.13 & -3.28 $\pm$ 0.11 & -3.25 $\pm$ 0.09 & -3.12 $\pm$ 0.07 \\
        10.50 -- 10.75 & -                & -                & -                & -3.49 $\pm$ 0.14 & -                & -3.54 $\pm$ 0.12 \\
        10.75 -- 11.00 & -                & -                & -                & -3.79 $\pm$ 0.19 & -                & -3.73 $\pm$ 0.14 \\
        \hline
        $\log_{10}(L_{IR}/L_{\odot})$ & 1.20 $\leq z <$ 1.70 & 1.70 $\leq z <$ 2.00 & 2.00 $\leq z <$ 2.50 & 2.50 $\leq z <$ 3.00 & 3.00 $\leq z <$ 4.20 & 4.20 $\leq z <$ 6.00  \\
        \hline
         9.75 -- 10.00 & -2.55 $\pm$ 0.02 & -                & -                & -                & -                & - \\
        10.00 -- 10.25 & -2.68 $\pm$ 0.02 & -2.55 $\pm$ 0.03 & -                & -                & -                & - \\
        10.25 -- 10.50 & -2.91 $\pm$ 0.03 & -2.67 $\pm$ 0.03 & -2.63 $\pm$ 0.02 & -                & -                & - \\
        10.50 -- 10.75 & -3.13 $\pm$ 0.04 & -2.85 $\pm$ 0.04 & -2.76 $\pm$ 0.03 & -2.76 $\pm$ 0.03 & -                & - \\
        10.75 -- 11.00 & -3.43 $\pm$ 0.06 & -3.02 $\pm$ 0.05 & -3.06 $\pm$ 0.04 & -2.84 $\pm$ 0.03 & -2.94 $\pm$ 0.02 & - \\
        11.00 -- 11.25 & -4.02 $\pm$ 0.12 & -3.40 $\pm$ 0.07 & -3.13 $\pm$ 0.04 & -3.04 $\pm$ 0.04 & -3.08 $\pm$ 0.02 & - \\
        11.25 -- 11.50 & -                & -3.64 $\pm$ 0.09 & -3.39 $\pm$ 0.05 & -3.21 $\pm$ 0.04 & -3.35 $\pm$ 0.03 & -3.77 $\pm$ 0.05 \\
        11.50 -- 11.75 & -                & -4.03 $\pm$ 0.14 & -3.68 $\pm$ 0.07 & -3.55 $\pm$ 0.06 & -3.69 $\pm$ 0.05 & -4.14 $\pm$ 0.07 \\
        11.75 -- 12.00 & -                & -4.29 $\pm$ 0.19 & -4.32 $\pm$ 0.15 & -3.94 $\pm$ 0.10 & -4.03 $\pm$ 0.07 & -4.27 $\pm$ 0.09 \\
        12.00 -- 12.25 & -                & -                & -                & -4.52 $\pm$ 0.19 & -4.19 $\pm$ 0.09 & -4.78 $\pm$ 0.15
        \botrule
    \end{tabular}
    \end{center}
    \begin{tabnote}
        {\textbf{Note}: Luminosity bin $\phi$ values are centred.}\tnp
    \end{tabnote}
\end{table*}

\begin{table*}
    \begin{center}
    \caption{CIGALE SF LF $\phi$ values.}
    \label{Tab: CG SF LF}
    \begin{tabular}{@{}ccccccc@{}}
        \toprule
        $\log_{10}(L_{IR}/L_{\odot})$ & 0.00 $\leq z <$ 0.30 & 0.30 $\leq z <$ 0.45 & 0.45 $\leq z <$ 0.60 & 0.60 $\leq z <$ 0.80 & 0.80 $\leq z <$ 1.00 & 1.00 $\leq z <$ 1.20 \\
        \hline
         8.50 --  8.75 & -1.74 $\pm$ 0.05   & - & - & - & - & - \\
         8.75 --  9.00 & -1.70 $\pm$ 0.04   & - & - & - & - & - \\
         9.00 --  9.25 & -1.83 $\pm$ 0.05   & -1.77 $\pm$ 0.03  & - & - & - & - \\
         9.25 --  9.50 & -1.83 $\pm$ 0.05   & -1.89 $\pm$ 0.04  & -1.90 $\pm$ 0.03  & - & - & - \\
         9.50 --  9.75 & -1.93 $\pm$ 0.06   & -2.00 $\pm$ 0.04  & -2.06 $\pm$ 0.04  & -1.90 $\pm$ 0.02  & - & - \\
         9.75 -- 10.00 & -2.02 $\pm$ 0.06   & -2.03 $\pm$ 0.05  & -2.02 $\pm$ 0.04  & -1.91 $\pm$ 0.02  & -2.06 $\pm$ 0.02  & - \\
        10.00 -- 10.25 & -2.14 $\pm$ 0.07   & -2.22 $\pm$ 0.06  & -2.17 $\pm$ 0.04  & -2.07 $\pm$ 0.03  & -2.08 $\pm$ 0.02  & -2.25 $\pm$ 0.03 \\
        10.25 -- 10.50 & -2.16 $\pm$ 0.08   & -2.23 $\pm$ 0.06  & -2.25 $\pm$ 0.05  & -2.10 $\pm$ 0.03  & -2.20 $\pm$ 0.03  & -2.27 $\pm$ 0.03 \\
        10.50 -- 10.75 & -2.32 $\pm$ 0.09   & -2.23 $\pm$ 0.06  & -2.39 $\pm$ 0.05  & -2.17 $\pm$ 0.03  & -2.30 $\pm$ 0.03  & -2.38 $\pm$ 0.03 \\
        10.75 -- 11.00 & -2.32 $\pm$ 0.09   & -2.51 $\pm$ 0.08  & -2.39 $\pm$ 0.05  & -2.28 $\pm$ 0.03  & -2.37 $\pm$ 0.03  & -2.46 $\pm$ 0.03 \\
        11.00 -- 11.25 & -2.64 $\pm$ 0.13   & -2.66 $\pm$ 0.09  & -2.57 $\pm$ 0.07  & -2.33 $\pm$ 0.04  & -2.43 $\pm$ 0.04  & -2.64 $\pm$ 0.04 \\
        11.25 -- 11.50 & -                  & -3.08 $\pm$ 0.15  & -2.90 $\pm$ 0.10  & -2.40 $\pm$ 0.04  & -2.50 $\pm$ 0.04  & -2.70 $\pm$ 0.04 \\
        11.50 -- 11.75 & -                  & -                 & -3.20 $\pm$ 0.14  & -2.72 $\pm$ 0.06  & -2.52 $\pm$ 0.04  & -2.68 $\pm$ 0.04 \\
        11.75 -- 12.00 & -                  & -                 & -                 & -3.28 $\pm$ 0.11  & -2.82 $\pm$ 0.06  & -2.95 $\pm$ 0.06 \\
        12.00 -- 12.25 & -                  & -                 & -                 & -3.79 $\pm$ 0.19  & -3.03 $\pm$ 0.07  & -3.43 $\pm$ 0.10 \\
        12.25 -- 12.50 & -                  & -                 & -                 & -                 & -3.76 $\pm$ 0.16  & -3.78 $\pm$ 0.15 \\
        \hline
        $\log_{10}(L_{IR}/L_{\odot})$ & 1.20 $\leq z <$ 1.70 & 1.70 $\leq z <$ 2.00 & 2.00 $\leq z <$ 2.50 & 2.50 $\leq z <$ 3.00 & 3.00 $\leq z <$ 4.20 & 4.20 $\leq z <$ 6.00  \\
        \hline
        10.50 -- 10.75 & -2.46 $\pm$ 0.02   & -2.64 $\pm$ 0.03  & - & - & - & - \\
        10.75 -- 11.00 & -2.58 $\pm$ 0.02   & -2.59 $\pm$ 0.03  & -2.77 $\pm$ 0.03  & - & - & - \\
        11.00 -- 11.25 & -2.61 $\pm$ 0.02   & -2.70 $\pm$ 0.03  & -2.79 $\pm$ 0.03  & -2.93 $\pm$ 0.03  & - & - \\
        11.25 -- 11.50 & -2.67 $\pm$ 0.02   & -2.58 $\pm$ 0.03  & -2.76 $\pm$ 0.03  & -2.84 $\pm$ 0.03  & - & - \\
        11.50 -- 11.75 & -2.65 $\pm$ 0.02   & -2.65 $\pm$ 0.03  & -2.76 $\pm$ 0.03  & -2.94 $\pm$ 0.03  & -3.17 $\pm$ 0.03 & - \\
        11.75 -- 12.00 & -2.82 $\pm$ 0.03   & -2.73 $\pm$ 0.03  & -2.78 $\pm$ 0.03  & -3.01 $\pm$ 0.03  & -3.30 $\pm$ 0.03 & -3.74 $\pm$ 0.05 \\
        12.00 -- 12.25 & -3.02 $\pm$ 0.04   & -2.87 $\pm$ 0.04  & -2.97 $\pm$ 0.03  & -3.17 $\pm$ 0.04  & -3.49 $\pm$ 0.04 & -4.03 $\pm$ 0.06 \\
        12.25 -- 12.50 & -3.32 $\pm$ 0.05   & -3.11 $\pm$ 0.05  & -3.17 $\pm$ 0.04  & -3.34 $\pm$ 0.05  & -3.82 $\pm$ 0.06 & -4.24 $\pm$ 0.08 \\
        12.50 -- 12.75 & -3.82 $\pm$ 0.09   & -3.39 $\pm$ 0.07  & -3.48 $\pm$ 0.06  & -3.54 $\pm$ 0.06  & -4.01 $\pm$ 0.07 & -4.91 $\pm$ 0.18 \\
        12.75 -- 13.00 & -                  & -4.21 $\pm$ 0.18  & -3.92 $\pm$ 0.10  & -4.27 $\pm$ 0.14  & -4.43 $\pm$ 0.12 & -4.91 $\pm$ 0.18
        \botrule
    \end{tabular}
    \end{center}
    \begin{tabnote}
        {\textbf{Note}: Luminosity bin $\phi$ values are centred.}\tnp
    \end{tabnote}
\end{table*}

\subsection{Fitting Functions}
\begin{itemize}
    \item \textcolor{red}{Since the LF parameters are not directly comparable using two different fitting functions, why not using the modified Schechter for all, ZFOURGE, AGN and SF?}
    \item \textcolor{Green}{Same as before}
\end{itemize}

We first construct the bolometric IR LF using the ZFOURGE dataset. Then, using CIGALE, we decompose the luminosity into contributions from SF regions and AGN, allowing us to investigate the evolution of these components separately.

To model LFs, one of the most widely used methods is the Schechter function \citep{schechter_analytic_1976}. This function is beneficial for describing the LF of galaxies because it can represent observed features, such as a power-law decline at the faint end and an exponential cutoff at the bright end. We employ the Schechter function to model the CIGALE SF LF as both \cite{fu_decomposing_2010} and \cite{wu_mid-infrared_2011} have shown that pure SF LFs fit better with a Schechter function. The Schechter function is mathematically represented by equation \ref{EQ: Shechter Function}:

\begin{equation} 
    \varphi(L) = \varphi^* \left(\frac{L}{L^*}\right)^{1-\alpha} \exp\left(-\frac{L}{L^*}\right) 
    \label{EQ: Shechter Function}
\end{equation}

Where $\varphi(L)$ is the number of galaxies per unit volume (number density), $\varphi^*$ is the characteristic normalisation factor, $L$ is the bolometric IR (8-1000$\mu$m) luminosity, $L^*$ is the characteristic luminosity, and $\alpha$ is the faint end slope \citep{schechter_analytic_1976}. The Schechter function, however, is not the only commonly used fitting function at mid- and far-IR wavelengths. The bright end slope of the Schechter function cannot be independently varied to fit a dataset better. We make use of a modified Schechter function known as the Saunders function (\citealp{saunders_60-mum_1990}; equation. \ref{EQ: Saunders Function}) to fit our ZFOURGE total and CIGALE AGN LFs:

\begin{equation} 
    \varphi(L) = \varphi^* \left(\frac{L}{L^*}\right)^{1-\alpha} \exp\left[-\frac{1}{2\sigma^2}\log_{10}^2\left(1+\frac{L}{L^*}\right)\right]
    \label{EQ: Saunders Function}
\end{equation}

Where the parameters are the same as the Schechter function (equation \ref{EQ: Shechter Function}), but with the introduction of $\sigma$ to vary the bright end slope. Our deep ZFOURGE data probes to fainter luminosities than often seen in the literature (e.g. \citealp{rodighiero_mid-_2010, gruppioni_herschel_2013}), thus better constraining the faint end of the LF. However, as ZFOURGE is designed to probe deeper into the universe, we lack brighter galaxies at lower redshifts. 

% \begin{equation}
%     D_{max} = \sqrt{\frac{L}{4 \pi F_{lim}}}
%     \label{EQ: Maximum Distance}
% \end{equation}

% As the bolometric IR (8-1000$\mu$m) luminosity has been independently estimated, equation \ref{EQ: Luminosity Distance} can be rearranged to calculate the maximum distance of a galaxy given its known luminosity, $L_{bol}$, and flux limit of the survey, $F_{lim}$ (equation \ref{EQ: Maximum Distance}). 



% As the bolometric IR luminosity of each galaxy is known, we calculate the maximum possible distance to each galaxy according to equation \ref{EQ: Maximum Distance} but using our new bolometric flux limits in each redshift bin. Figure \ref{Fig: Lum vs z} shows the bolometric luminosity (left) and maximum observable distance across redshift evolution (right) according to the flux limits each in redshift bin. The maximum observable distance is capped to the end of the redshift bin. The maximum observable volume of each galaxy is calculated from the minimum and maximum comoving distance of the redshift bin that the galaxy appears in, corrected for the survey area in square degrees (equation \ref{EQ: Maximum Volume}). For ZFOURGE, the total sky area observed across the 3 fields is 0.1111 square degrees \citep{straatman_fourstar_2016}.

% There are different equations used to model the LF of galaxies and BHs. One commonly used approach is the $1/V_{max}$ method \citep{schmidt_space_1968}, which we implement in this work. The formula for the $1/V_{max}$ method is given by formula \ref{EQ: Vmax} below, where $V_{max}$ represents the comoving volume of the $i$-th source measured in the corresponding redshift bin between $z_{min}$ and $z_{max}$ and $\Delta$ log(L) is the width of the luminosity bin.

% Using the UVJ selection criteria, the star-forming LF is generated, but not plotted in Figure \ref{Fig: Lum vs z} because it is too similar to the normal IR LF \color{red} describe UVJ here or later? (currently in discussion) \color{black}.

% We fix $\alpha=1.2$ for the Schechter function across all redshift bin function fits and leave $\varphi^*$ and $L^*$ as free parameters. 

% We fix $\sigma=0.9$ in the first two redshift bins and $\sigma=0.6$ in all other bins.

% However, the Schechter function is not the best-fitting function at mid- and far-IR wavelengths \color{red} (Cite) \color{black}. The bright end slope of the Schechter function is too steep to fit our data (see figure \ref{Fig: Bolometric IR LF}), in agreement with the literature \citep{rodighiero_mid-_2010, gruppioni_herschel_2013}. To better fit our data, we make use of a modified Schechter function known as the Saunders function

% an accurate picture of galaxy distribution and evolution is acquired by splitting the data into redshift bins. The differences in the bins provide insight into how galaxies have evolved through time and point to periods of increased or decreased activity. ZFOURGE allows us to look at the infrared (IR) LF of galaxies with increased depth, extending the LF to potentially redshift of $z \approx 6$.

% For example, galaxies above $10^{12}$ $L_{\odot}$ appearing in the first redshift bin $0.00 \leq z < 0.30$ have a maximum distance and volume corresponding to $z=0.30$, even though such galaxies could potentially be detected up to $z=4$ with the same flux limit.

% Galaxy luminosity functions (LF) are statistical distributions that describe the spatial density of galaxies at different luminosities and are a fundamental tool for quantifying their evolution across cosmic time scales \citep{han_evolution_2012, dai_mid-infrared_2009, wylezalek_galaxy_2014}. By incorporating the AGN LF with SED decomposition software and comparing with the SF LF, we can probe the evolution of galaxies and the co-evolution of AGN to gain deeper insights into the mechanisms driving galaxy formation.

% At our lowest redshift bins (\ref{Fig: Bolometric IR LF}), we do not have comparatively high luminosity bins to fit the bright end of the LF. We fix $\sigma$ with values matching the bright end of the literature (see section \ref{Sec: Parameter Evolution} for a discussion on parameter evolution). We opt to fix $\alpha=1.3$ for the ZFOURGE LF and $\alpha=1.2$ our CIGALE LFs as these values better fit our data and are in good agreement with literature. We leave $L^{*}$ and $\phi^{*}$ as free parameters as done in the literature as well. The evolution of $L^{*}$ and $\phi^{*}$ will be discussed in section \ref{Sec: Parameter Evolution} where we show the evolution in figure \ref{Fig: Param Evo} and values in tables \ref{Tab: Param Evo ZF} and \ref{Tab: Param Evo AGN} for ZFOURGE and CIGALE respectively.