\section{Introduction}

Galaxy luminosity functions (LF) are statistical distributions that describe the spatial density of galaxies at different luminosities and are a fundamental tool for quantifying their evolution across cosmic time scales \citep{han_evolution_2012, dai_mid-infrared_2009, wylezalek_galaxy_2014}.

By splitting the data into redshift bins, an accurate picture of galaxy distribution and evolution is acquired. The differences in the bins provide insight into how galaxies have evolved through time and point to periods of increased or decreased activity. ZFOURGE allows us to look at the infrared (IR) LF of galaxies with increased depth, extending the LF to potentially redshift of $z \approx 6$.

To calculate LFs, one of the most widely used methods is the Schechter function \citep{schechter_analytic_1976}. This function is beneficial for describing the LF of galaxies because it accurately represents observed features such as a power-law decline at the faint end and an exponential cutoff at the bright end. 

The Schechter function is given as:

\begin{equation} 
    \varphi(L) = \varphi^* \left(\frac{L}{L^*}\right)^\alpha \exp\left(-\frac{L}{L^*}\right) 
    \label{EQ: Shechter Function}
\end{equation}

Where $\varphi(L)$ is the number of galaxies per unit volume (number density) from $L$ to $L+dL$. $L$ is the luminosity. $\varphi^*, L^*,$ and $\alpha$ are fitting parameters specific to the data set \citep{schechter_analytic_1976}. As we fit our data in log space, we modify equation \ref{EQ: Shechter Function} to offset the positioning to best fit our data with:

\begin{equation} 
    \varphi(L) = \varphi^* \left(10^{-0.4*(1-\alpha)(L_{*}-L)}\right) \exp\left(-10^{-0.4(L_{*}-L)}\right)
    \label{EQ: Shechter Function}
\end{equation}

There are different equations used to model the LF of galaxies and BHs. One commonly used approach is the $1/V_{max}$ method \citep{schmidt_space_1968, enia_new_2022}, which we implement in this work. The formula for the $1/V_{max}$ method is given by formula \ref{EQ: Vmax} below, where $V_{max}$ represents the co-moving volume of the $i$-th source measured in the corresponding redshift bin between $z_{min}$ and $z_{max}$.

\begin{equation} \label{EQ: Vmax}
    \phi(L,z) = \frac{1}{\Delta \log L}\sum_i \frac{1}{V_{max}}
\end{equation}

In section 2 we ... In section 3 we ... In section 4 we ... In section 5 we ... In section 6 we ...

We adopt a cosmology of $H_0 = 70$ km $\mathrm{s^{-1}\ Mpc^{-1}}$, $\Omega_m=0.7$, and $\Omega_\Lambda=0.3$.

\color{red}
Research by \cite{wu_mid-infrared_2011} has shown that the UV and optical wavelengths closely follow a Schechter function. In contrast, the IR wavelengths have a shallower exponential which is inconsistent with a Schechter function. \cite{fu_decomposing_2010} proposed that this difference is due to the AGN contribution to the IR Galaxy LF. Even though this research primarily focuses on IR AGN LFs, it is essential to compare the IR Galaxy LF differences which this paper focuses on. In the next paper, we dive into IR AGN LFs.

\color{LimeGreen}
Galaxy luminosity distribution and the environment have often been used to provide strong constraints of theories of galaxy evolution \cite{biviano_spitzer_2011}

powerful constraints on how galaxies evolve in relation to their environment are expected to be obtained from the analysis of the galaxy IR LFs. \cite{biviano_spitzer_2011}

IR data can give information only for dusty massive galaxies and are limited especially at high redshifts \cite{katsianis_evolution_2017}

alpha, the faint end slope of the galaxy LF, does not evolve significantly with redshift. \cite{wylezalek_galaxy_2014}

Confirm that evolution on both luminosity and density is required to explain the difference in the LFs at different redshifts. \cite{wu_mid-infrared_2011}

Extensively studied at low redshift. \cite{wu_mid-infrared_2011}

SF may also heat the dust in galaxies and lead to excess emission in the longer wavelengths \cite{oconnor_luminosity_2016}

The SF fraction is also found to be a function of luminosity/redshift, decreasing as luminosity or redshift increases, while the trend is more obvious in the MIR, suggesting that the MIR wavelength is more sensitive to the presence of AGNs \cite{wu_mid-infrared_2011}

Star forming galaxies reemit a significant portion of the ultraviolet and optical radiation absorbed by dust in the infrared regime. \cite{symeonidis_agn_2021, fu_decomposing_2010}

The obscured or absorbed optical, UV and X-ray radiation will be re-emitted in the IR. \cite{han_evolution_2012, brown_infrared_2019}

The IR-bright, dust obscured galaxy population is crucial to understanding galaxy formation and evolution. \cite{gruppioni_modelling_2011}

IR bright galaxies emit the bulk of their energy as dust-reprocessed light generated by dusty SF or accretion onto the supermassive black holes referred to hereafter as active galactic nuclei. \cite{wu_mid-infrared_2011}

Absorption by dust reprocesses the shorter wavelength radiation from the accretion disk into mid/far-IR continuum radiation. \cite{assef_mid-ir-_2011}

\color{Goldenrod}
The evolution of star formation activity in galaxies over cosmic history, and the physical process which may drive and limit such activity, have been the subject of intense observational and theoretical study in recent years. \cite{grazian_galaxy_2015}