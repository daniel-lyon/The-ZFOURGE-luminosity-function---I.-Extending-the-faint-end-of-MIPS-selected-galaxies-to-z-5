\section{Data}

All the necessary data for analysis is readily available for download from the ZFOURGE \citep{straatman_fourstar_2016} database at VizieR\footnote{ZFOURGE Data: \url{http://vizier.cds.unistra.fr/viz-bin/VizieR?-source=J/ApJ/830/51&-to=2}}. The dataset includes all three 11x11 arcminute fields under the ZFOURGE survey: the Chandra Deep Field South (CDFS) located at RA: 10h 00m 31s and DEC: $+02^\circ 17' 03"$ \citep{giacconi_chandra_2002}; the Cosmic Evolution Survey (COSMOS) located at RA: 03h 32m 27s and DEC: $-27^\circ 45' 52"$ \citep{scoville_cosmic_2007}; and the CANDELS Ultra Deep Survey (UDS) located at RA: 02h 17m 15s and DEC: $-05^\circ 11' 53"$ \citep{lawrence_ukirt_2007}.

\color{red}
In order to select the maximum number of candidate galaxies and minimise errors, the only criteria to be applied is the Use=1 category. This ensures that all selected sources are known galaxies well-defined by redshift, distance, magnitude, and other characteristics. ZFOURGE can stretch up to redshift $z \simeq 5$, allowing us to study a wide range of epochs. This data range enables us to observe the evolution of galaxies during some of the most critical cosmic periods, specifically around $1 < z < 3$ \citep{gruppioni_modelling_2011, wylezalek_galaxy_2014} where luminous density is at its maximum before declining \citep{assef_mid-ir-_2011}.