\lefttitle{Publications of the Astronomical Society of Australia}
\righttitle{D. J. Lyon \textit{et al.}}

\jnlPage{1}{4}
\jnlDoiYr{2021}
\doival{10.1017/pasa.xxxx.xx}

\articletitt{Research Paper}

\title{Decomposing Infrared Luminosity Functions into Star-Forming and AGN Components using CIGALE}

\author{\gn{Daniel J.} \sn{Lyon}$^{1}$,  
        \gn{Michael J.} \sn{Cowley}$^{1,2}$,
        \gn{Oliver} \sn{Pye}$^{1}$,
        and \gn{Andrew M.} \sn{Hopkins}$^{3}$}

\affil{
$^1$School of Chemistry and Physics, 2 George St, Queensland University of Technology, Brisbane, QLD 4000, Australia \\ 
$^2$Centre for Astrophysics, University of Southern Queensland, West Street, Toowoomba,
QLD 4350, Australia\\
$^3$School of Mathematical and Physical Sciences, 12 Wally’s Walk, Macquarie University, NSW 2109, Australia}

\corresp{Daniel Lyon, Email: daniellyon31@gmail.com}

\citeauth{Author1 C and Author2 C, an open-source python tool for simulations of source recovery and completeness in galaxy surveys. {\it Publications of the Astronomical Society of Australia} {\bf 00}, 1--12. https://doi.org/10.1017/pasa.xxxx.xx}

\history{(Received xx xx xxxx; revised xx xx xxxx; accepted xx xx xxxx)}

\begin{abstract}
    This study presents a comprehensive analysis of the infrared (IR) luminosity functions (LF) of star-forming (SF) galaxies and active galactic nuclei (AGN) using data from the ZFOURGE survey. We employ CIGALE to decompose the spectral energy distribution (SED) of galaxies into SF and AGN components to investigate the co-evolution of these processes at higher redshifts and fainter luminosities. Our CIGALE-derived SF and AGN LFs are generally consistent with previous studies, with an enhancement at the faint end of the AGN LFs. We attribute this to CIGALE's capability to recover low-luminosity AGN more accurately, which may be underrepresented in other works. As anticipated, the CIGALE SF LFs are best fit with a Schechter function, whereas the AGN LFs align more closely with a Saunders function. We find evidence for a significant evolutionary epoch for AGN activity at $z \approx 1$, comparable to the peak of cosmic star formation at $z \approx 2$, which we also recover well. Based on our results, the gas supply in the early universe favoured the formation of brighter star-forming galaxies until $z=2$, after which the gas for SF becomes increasingly exhausted. Conversely, AGN activity peaked earlier and declined more slowly until $z \approx 1$, suggesting a possible feedback scenario in which $2.5-3$ Gyrs offset the evolution of SF and AGN activity. 
\end{abstract}

\begin{keywords}
galaxies: luminosity function, mass function; cosmology: observations; infrared: galaxies; galaxies: evolution
\end{keywords}

\maketitle

% This reflects an AGN-SF feedback scenario in which increased AGN activity follows after $\approx$ 2 Gyrs of SF decline. AGN activity peaks sometime between $2 < z < 3$, likely earlier than SF galaxies. These results indicate a negative correlation with a time lag between AGN and SF galaxies. As SF increases, AGN decreases. When the supply of gas to form stars is exhausted, SF declines and the residual gas is collected at the central SMBH to become an AGN and activity increases.

% We make use of 8,800 AGN (11.9\% of our reduced sample) with a significant AGN fraction ($\mathcal{F}_{AGN} > 0.1$).