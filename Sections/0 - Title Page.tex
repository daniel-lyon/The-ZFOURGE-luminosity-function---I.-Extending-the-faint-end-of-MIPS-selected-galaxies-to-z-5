\lefttitle{Publications of the Astronomical Society of Australia}
\righttitle{D. J. Lyon \textit{et al.}}

\jnlPage{1}{4}
\jnlDoiYr{2021}
\doival{10.1017/pasa.xxxx.xx}

\articletitt{Research Paper}

\title{Infrared Luminosity Functions Decomposed into Star-Forming and AGN Components using CIGALE}

\author{\gn{Daniel J.} \sn{Lyon}$^{1}$,  
        \gn{Michael J.} \sn{Cowley}$^{1}$,
        \gn{Oliver} \sn{Pye}$^{1}$,
        and \gn{Andrew M.} \sn{Hopkins}$^{2}$}

\affil{
$^1$School of Chemistry and Physics, 2 George St, Queensland University of Technology, Brisbane, QLD 4000, Australia \\ 
$^2$School of Mathematical and Physical Sciences, 12 Wally’s Walk, Macquarie University, NSW 2109, Australia
}

\corresp{Daniel Lyon, Email: daniellyon31@gmail.com}

\citeauth{Author1 C and Author2 C, an open-source python tool for simulations of source recovery and completeness in galaxy surveys. {\it Publications of the Astronomical Society of Australia} {\bf 00}, 1--12. https://doi.org/10.1017/pasa.xxxx.xx}

\history{(Received xx xx xxxx; revised xx xx xxxx; accepted xx xx xxxx)}

\begin{abstract}
    This study presents a detailed analysis of the infrared (IR) luminosity functions (LF) of Star-Forming (SF) galaxies and Active Galactic Nuclei (AGN). Using data from the ZFOURGE survey, we utilise CIGALE to decompose the spectral energy distribution (SED) of galaxies into SF and AGN components to investigate their co-evolution. We make use of 7,287 AGN (32\% of our reduced sample) with a significant AGN fraction ($F_{AGN} > 0.1$). We find good agreement with our SF and AGN LFs to the literature, with only our AGN LFs elevated at the faint end which we attribute to CIGALE recovering fainter AGN than previous studies. As predicted, the SF LF is better fit with a Schechter function whereas the AGN LF is better fit with a Saunders function. We find evidence for a significant evolutionary epoch for AGN activity at $z \approx 1$ similar to the peak of cosmic star formation at $z \approx 2$ which we also recover well. Based on our results, the supply of gas in the early universe increasingly formed brighter SFG until $z=2$, after which the gas for SF becomes increasingly exhausted. AGN were brightest in the early universe and declined until $z=1$ as gas was predominately utilised by SF. As SF activity declined at $z=2$, AGN activity increased from $0.5<z<1$. This represents an AGN-SF Feedback scenario in which the rise and fall of each is offset by $\approx$ 2-5 --- 3 Gyrs. 
\end{abstract}

\begin{keywords}
galaxies: luminosity function, mass function; cosmology: observations; infrared: galaxies; galaxies: evolution
\end{keywords}

\maketitle

% This reflects an AGN-SF feedback scenario in which increased AGN activity follows after $\approx$ 2 Gyrs of SF decline. AGN activity peaks sometime between $2 < z < 3$, likely earlier than SF galaxies. These results are indicative of a negative correlation with a time lag between AGN and SF galaxies. As SF increases, AGN decreases. When the supply of gas to form stars is exhausted, SF declines and the residual gas is collected at the central SMBH to become an AGN and activity increases.