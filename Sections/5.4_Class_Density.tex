\subsection{Space Density Evolution}
\label{Sec: Class Density}

In this section, we inspect the evolution of different luminosity classes by visualising the number density of each luminosity bin across redshift (figure \ref{Fig: Class Evo}). By incorporating the class density evolution into our analysis, a better picture of the evolution of galaxies and the co-evolution of AGN can be ascertained. When possible, the $\phi$ values are taken from the existing luminosity bins\footnote{$\phi$ values ending in X.875 and X.375 $\log_{10}(L_{\odot})$ are skipped. Shown bins are 0.25 $\log_{10}(L_{\odot})$ wide.}. Otherwise, $\phi$ values are calculated from the best-fitting LF. The class density evolution (figure \ref{Fig: Class Evo}) can be thought of as the transpose of the LF (figures \ref{Fig: Bolometric IR LF} and \ref{Fig: LF Filled}). The LF and class density are complementary because they allow us to view the number density as an evolution with luminosity and redshift, respectively. 

\begin{table}[h]
    \begin{center}
    \caption{Luminosity density as a function of redshift. Units are in log($\rho_{IR}$) [$L_{\odot}$ Mpc$^{-3}$]. $\rho_{IR}$ values are centered on the redshift bin.}
    \label{Tab: SFRD}
    \begin{tabular}{@{}cccc@{}}
        \toprule
        $z$ & ZFOURGE & CIGALE SF & CIGALE AGN \\
        \hline
        0.30 $\leq z <$ 0.45 & 7.98 $\pm$ 0.81 & 8.34 $\pm$ 0.66 & 7.16 $\pm$ 0.85 \\
        0.45 $\leq z <$ 0.60 & 8.25 $\pm$ 0.70 & 8.36 $\pm$ 0.40 & 7.24 $\pm$ 0.44 \\
        0.00 $\leq z <$ 0.30 & 8.40 $\pm$ 0.95 & 8.44 $\pm$ 0.42 & 7.63 $\pm$ 0.27 \\
        0.60 $\leq z <$ 0.80 & 8.69 $\pm$ 0.59 & 8.83 $\pm$ 0.44 & 7.69 $\pm$ 0.26 \\
        0.80 $\leq z <$ 1.00 & 8.77 $\pm$ 0.73 & 8.77 $\pm$ 0.56 & 7.81 $\pm$ 0.36 \\
        1.00 $\leq z <$ 1.20 & 8.70 $\pm$ 0.68 & 8.77 $\pm$ 0.49 & 7.79 $\pm$ 0.55 \\
        1.20 $\leq z <$ 1.70 & 9.03 $\pm$ 0.80 & 9.25 $\pm$ 0.44 & 8.19 $\pm$ 0.58 \\
        1.70 $\leq z <$ 2.00 & 9.27 $\pm$ 0.34 & 9.56 $\pm$ 0.37 & 8.50 $\pm$ 0.68 \\
        2.00 $\leq z <$ 2.50 & 9.11 $\pm$ 0.54 & 9.53 $\pm$ 0.32 & 8.52 $\pm$ 0.79 \\
        2.50 $\leq z <$ 3.00 & 9.13 $\pm$ 0.67 & 9.35 $\pm$ 0.39 & 8.58 $\pm$ 1.03 \\
        3.00 $\leq z <$ 4.20 & 9.34 $\pm$ 0.65 & 9.04 $\pm$ 0.48 & 8.57 $\pm$ 0.51 \\
        4.20 $\leq z <$ 6.00 & 9.07 $\pm$ 0.34 & 8.65 $\pm$ 0.36 & 8.16 $\pm$ 0.66 
        \botrule
    \end{tabular}
    \end{center}        
\end{table}

In figure \ref{Fig: Class Evo}, we present IR luminosity classes as low as $L_{IR}=10^{8.5}\ L_{\odot}$. We find that the space density of ZFOURGE LIRGs and ULIRGs have been consistently declining since at least $z=2$, and likely even earlier for ULIRGs. Galaxies fainter than LIRGs (FIRGs, $L_{IR} < 10^{11} L_{\odot}$) evolve differently, beginning to decline at a lower redshift than their brighter luminosity counterparts. The redshift at which galaxies begin declining in number density is related to their luminosity. ZFOURGE galaxies fainter than $L_{IR} < 10^{9}\ L_{\odot}$ appear to be increasing in number density across all of cosmic time and have yet to begin declining. We find similar agreement in the literature with \cite{rodighiero_mid-_2010} and \cite{gruppioni_herschel_2013} with our results mostly in agreement. We attribute the differences to slight variations in the classes and methods within. FIRGs dominate the ZFOURGE LD from $0<z<1.2$, declining from 74\% to 47\%. LIRGs dominate from $1.2<z<3$, remaining steady with 43\% to 51\% contribution. At $z>3$, ULIRGs dominate LD, increasing rapidly from 38\% to 74\%. In the highest redshift bin, FIRG contribution drops to 5\%

CIGALE SF galaxies evolve differently from ZFOURGE, although similar contributions to the LD are seen. FIRGs dominate LD density from $0<z<1.0$, remaining roughly constant between 49\% to 63\%. LIRGs only dominate LD from $1.0<z<1.2$ with 45\% contribution. ULIRGs dominate LD from $z>1.2$ onwards, with contribution increasing from 48\% to 75\%. Figure \ref{Fig: Class Evo} shows that SF LIRGs evolve similarly to FIRGs at $z>2$. However, the estimated $\phi$ values show an increasing number density with decreasing luminosity for all FIRGs. A possible evolution scenario is theorised for SFG: all evolve similarly from high redshift to $z \approx 2$, increasing in number density from high redshift. At and below $z \approx 2$, the brightest FIRG number densities begin to peak and decline earlier than fainter FIRG counterparts, which have yet to start declining. LIRGs decline faster and earlier than FIRGs, and ULIRGs decline faster and sooner than LIRGs. This reflects a \textit{downsizing} scenario in which brighter galaxies peak in number density at higher redshift \citep{merloni_synthesis_2008, wylezalek_galaxy_2014, fiore_agn_2017}.

CIGALE AGN again evolve differently. Faint IR AGN dominate LD from $0 \leq z < 1.7$ and luminous IR AGN dominate LD from $z>1.2$ onwards. Ultra-luminous IR AGN never dominates. The difference between FIRGs and LIRGs is stark. There is a clear, systematic shift in the peak number density with luminosity class. The \textit{downsizing} seen in SF galaxies is more pronounced in AGN. As AGN luminosity increases, the number density of AGN peaks at higher redshifts and declines earlier than their fainter luminosity counterparts. When comparing the \textit{downsizing} effect between SFGs and AGN, it is unmistakable that galaxies with a luminous AGN decline faster and earlier than equally bright SFG counterparts.

% despite making up 98\% of sources at $z=0$, FIRGs only contribute 38\% of the total luminosity.

% LIRGs contribute only 2\% of the number density but make up 62\% of the total luminosity at $z=0$. At $z=2$, FIRGs contribute 46\% and 8\% of the total number density and luminosity respectively. LIRGS 45\% and 42\% respectively. ULIRGs, despite making up just under 10\% of the number density at $z=2$, contribute 50\% of the total infrared luminosity. FIRGs number and luminosity contribution drops to 0\% above $z=3$. In our final redshift bin, LIRG number and luminosity contribution is 21\% and 5\% respectively. ULIRG number and luminosity contribution is 79\% and 95\% respectively. However, these results only account for real measured galaxies and do not include the estimated galaxies. Otherwise, FIRGs would dominate the number density across all cosmic time.

% These results indicate that the brightest galaxies have been declining in number very shortly after the Big Bang. More luminous galaxies decline faster and at an earlier time period. Fainter galaxies take much longer to begin declining, if the faintest do at all.

% It is clear that ULIRGs decline much earlier than LIRGs, and LIRGs decline earlier than FIRGs. This presents a clear trend of brighter objects declining in number density more rapily than fainter galaxies.

% Across our data, we only have two luminosity bins between $10^{8.5} < L_{IR} < 10^{9}$ in the lowest redshift bin only, so results for this luminosity class could be improved in future work. Nevertheless, our results indicate very little number density evolution across all redshifts measured in our lowest luminosity class $L_{IR}=10^{8.5} L_{\odot}$. At $L_{IR}=10^{9} \& 10^{9.5} L_{\odot}$ there is again very little evolution until $z \approx 0.5$ when the number density again beings to decline. These results across a wider range of luminosities and redshifts previously measured in the literature confirm a \textit{downsizing} effect of all luminosity classes, of which the brightest galaxies begin declining before fainter ones.

% but only dominate the total luminosity in two redshift bins, $0.3 \leq z < 0.45$ and $0.6 \leq z < 0.8$. LIRGs dominate number density from $2 \leq z < 4$, but dominate luminosity from $0 \leq z < 0.3$, $1.0 \leq z < 1.7$, and $2.0 \leq z < 3.0$. From $0.45 \leq z < 0.6$, ULIRGs, despite only contributing 0.45\% to the number density, dominate the luminosity contribution with 42\%. ULIRGs only dominate the number density in our final redshift bin, and dominate luminosity from $1.7 \leq z < 2.0$, and $z>3$.

% \textcolor{red}{Our highest redshift bin (centered on $z=5.1$) is likely incomplete, and so the $\phi$ values here should be taken as a lower limit. In that case, it seems likely that LIRGs and ULIRGs have been declining in number density since at least $z=5$ when the universe was barely more than 1 Gyr old.}