\section{Conclusion}

We utilised ZFOURGE data as inputs for CIGALE to separate the SF and AGN components of the infrared spectral energy distribution to create and analyse their respective luminosity functions. Our results are summarised as follows:

\begin{enumerate}[(i)]
    \item We first generate LFs of ZFOURGE galaxies from a sub-sample of 18,373 galaxies that accounted for incompleteness and bias across various redshift bins ranging from $0<z<6$.
    
    \item After CIGALE decomposition, we find 21,390 sources (28.9\% of the sample) with a significant AGN fraction ($\mathcal{F}_{AGN} > 0.1$). This was reduced to 16,850 sources (22.8\%) after an 80\% flux completion cut. In context with AGN dominant sources removed in ZFOURGE by \cite{cowley_zfourge_2016}, simple MIR colour-colour selection diagnostics are insufficient to identify low luminosity AGN, which we show significantly influences galaxy samples.
    
    \item We use a Saunders function to fit CIGALE SF LFs, although a Schechter function may fit too, as predicted by \cite{wu_mid-infrared_2011, fu_decomposing_2010}. Brighter SF galaxies at higher redshifts are required to confirm. 

    \item A peak and turnover in the evolution of CIGALE SF $L^{*}$ is found at $z \approx 2$ coinciding with the peak of cosmic star formation. Similarly, SF LD is also seen to peak and decline above $z\approx2$. This agrees with established literature and provides good evidence supporting our CIGALE results. Importantly, no turnover in ZFOURGE LD was found, highlighting the importance of SED decomposition.
    
    \item A peak and turn over in $\phi^{*}$ is seen for CIGALE AGN at $z \approx 2$ with $L^{*}$ continuing to decline. Although the AGN LD declines from $z<2$, above $z>2$ does not show a defined peak. The peak of AGN LF likely lies between $z\approx2.5-3.5$. \cite{delvecchio_tracing_2014} finds a similar trend at the same time for their AGN $\phi^{*}$, but declines below $z<1$. \cite{katsianis_evolution_2017} also shows AGN feedback significantly affects SF at $z<1$. We have found reliable evidence that $z\approx2$ is a significant evolutionary epoch for AGN evolution. 
    
    \item In context, $L^{*}_{SF}$ peaks at the same time as $\phi^{*}_{AGN}$ while $\phi^{*}_{SF}$ increases across all cosmic time. These results show fainter SF galaxies becoming increasingly common in the local universe while brighter SF galaxies trend towards extinction.  

    \item The space density evolution of luminosity classes for SFG and AGN are very different. SF LIRGs and ULIRGs increase in number density and peak at $z \approx 2$ before rapidly declining to the present day. SF FIRGs evolve similarly across fainter luminosities and peak at a lower redshift. The faintest SF FIRGs have yet to peak in number density. AGN number densities peak much earlier in the universe than SF counterparts. Luminous and ultraluminous AGN have declined since at least $z \approx 3$ and likely even earlier. There is a strong correlation between peak AGN density and luminosity. Fainter AGN peak much later in the universe and have declined since $z \approx 2$. This shows a clear \textit{downsizing} effect which is more prominent in AGN.

    \item From high redshift until $z\approx2$, SFGs form with increasing brightness and consume a significant fraction of the available gas supply. Below $z<2$, the faintest SFG grow in number, but not as big and bright at $z<2$ as the gas supply becomes increasingly exhausted. This is reflected in our class evolution results and is the likely reason behind the \textit{downsizing} effect.

    \item The brightest AGN are already in decline at the beginning of the universe. These AGN decline at a consistent rate until $z\approx2$. Conversely, the faintest AGN increase in number density until $z\approx2$. Below $z=2$, all AGN show a rapid decline. At $z\approx1$, the brightest AGN show signs of increasing in number. This increase appears to last until $z\approx0.5$ before declining again. This reflects a period where the available gas supply shifted to fueling AGN growth.
\end{enumerate}

These results indicate a scenario where the available gas supply favoured SFG from high redshift until $z\approx2$ when the supply becomes increasingly exhausted. Below $z<2$, the remaining gas predominantly formed fainter SFG, with bright SFG rapidly declining to near extinction by $z\approx1$ when number densities stabilise. By $z=1$, the brightest AGN are almost as equally common as the brightest SFG. Given that AGN of a particular brightness decline sooner than equally bright SFG counterparts, it is probable that AGN positively influences the growth of SF.

\section*{Acknowledgements}
This research was supported by an Australian Government Research Training Program (RTP) Scholarship. Thank you to the anonymous referee, who provided valuable comments that improved this research. Thank you to our colleague Vanessa Porchet for using CIGALE to run a reduced-band analysis. Thank you to Vera Delfavero for helpful comments.

\section*{Data Availability}
Python notebooks and scripts that analysed the data are available on GitHub at \url{https://github.com/daniel-lyon/MPhil-Code}